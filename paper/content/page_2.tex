\section{Results and Analysis}

\subsection{Quantitative Results}

Die Technologie der Augmented Reality (“Erweiterte Realität”, AR) findet in der Archäologie Anwendung, insbesondere im Bereich des erweiterten Tourismus. Sie zielt darauf ab, die Erfahrung des Besuchs historischer Stätten zu verbessern, indem virtuelle Rekonstruktionen über die reale Umgebung gelegt werden. Dies wird durch den Einsatz von Geräten wie Head Mounted Displays (HMDs) oder mobilen Geräten erreicht, die virtuelle Objekte über Live-Videoübertragungen legen. Bemerkenswerte Beispiele für diese Technologie sind “ARCHEOGUIDE” und das Projekt “Cultural Heritage Experiences through Socio personal Interactions and Storytelling”.

Über den virtuellen Tourismus hinaus kann AR auch helfen, archäologische Stätten aus einer phänomenologischen Perspektive zu erkunden. Durch die Überlagerung von geografischen Daten aus einem Geografischen Informationssystem (GIS) mit der realen Welt wird eine verkörperte Erfahrung der Landschaft ermöglicht. Dieses Konzept wird anhand eines Projekts in einer bronzezeitlichen Siedlung in Cornwall, Grossbritannien, demonstriert, bei dem eine massstabsgetreue Rekonstruktion der Rundhäuser der Siedlung mit Hilfe einer Spiele-Engine, GIS-Software, einem AR-Plugin und benutzerdefinierten Skripten über die reale Landschaft gelegt wurde. 
\cite{Archaeology}

\subsection{Qualitative Assessment}

Die Technologie der Augmented Reality (“Erweiterte Realität”, AR) wird in vielen Anwendungen eingesetzt, wobei die Übersetzung eine ihrer Anwendungen ist. Ein Beispiel, mit dem viele Menschen vielleicht vertraut sind, aber nicht wissen, dass dort eine AR-Technologie verwendet wird, ist die Word Lens Mobile App, die von Google übernommen und in Google Translate integriert wurde. Mit dieser App können die Nutzer einfach ihre Smartphone-Kamera auf Texte wie Schilder oder Speisekarten richten und bekommen sofort eine Übersetzung neben dem Originaltext auf dem Bildschirm angezeigt. Damit entfällt die Notwendigkeit, ein Foto zu machen oder den Text manuell in die App einzugeben, was für Reisende und Sprachschüler auf der ganzen Welt, die Sprachen verstehen und mit ihnen interagieren wollen, äusserst praktisch ist. \cite{QuestVisual_2010}


\subsection{Comparison to Human Performance}

Augmented Reality (AR) hat das Potenzial, die Bildung durch die Schaffung immersiver Lernumgebungen erheblich zu verändern. Diese Umgebungen ermöglichen es den Lernenden, mit Objekten und Informationen in der realen Welt zu interagieren, was eine praktische und fesselnde Lernerfahrung ermöglicht. AR kann nahtlos in Aktivitäten wie Simulationen, Spiele und Prüfungen integriert werden, um die Entwicklung von kognitiven, emotionalen und physischen Fähigkeiten zu unterstützen. Darüber hinaus bietet AR personalisierte und adaptive Lernerfahrungen, die auf die spezifischen Bedürfnisse der Lernenden zugeschnitten sind und das Interesse, die Motivation und die Zusammenarbeit unter den Schülern fördern. Die pädagogischen Vorteile von AR sind zahlreich, da sie das Behalten von Wissen, die Problemlösungsfähigkeit und die Entscheidungsfindung verbessern und gleichzeitig Kreativität und Innovation fördern. \cite{Wu2013CurrentSO}

Allerdings muss man sich darüber im Klaren sein, dass die Integration von AR in den Unterricht auch mit Herausforderungen verbunden ist. Zu diesen Herausforderungen gehören die Notwendigkeit einer geeigneten Hardware- und Software-Infrastruktur sowie die fehlende Standardisierung verschiedener Plattformen oder Kompatibilitätsprobleme zwischen Systemen oder Geräten. Darüber hinaus besteht die Gefahr, dass die Lernergebnisse beeinträchtigt werden, wenn die Implementierung von AR nicht durchdacht konzipiert oder effektiv durchgeführt wird. Daher müssen Pädagogen und Designer verschiedene Ansätze sorgfältig abwägen und dabei die ethischen Bedenken im Zusammenhang mit dem Einsatz von AR im Bildungsbereich im Auge behalten. Auf diese Weise können sie Strategien und Leitlinien entwickeln, die den Nutzen der erweiterten Realität im Bildungsbereich maximieren.\cite{Wu2013CurrentSO}


\subsection{Statistical Analysis}

Einer der Hauptvorteile der Augmented-Reality-Technologie (AR) in der Videospielindustrie besteht darin, dass sie es den Nutzern ermöglicht, mit synthetischen Objekten, Personen und Umgebungen zu interagieren, die über reale Umgebungen gelegt werden, und ihre Erfahrungen in diesen Umgebungen mit computergenerierten Bildern und Tönen zu erweitern. Dies verbessert das Spielerlebnis und schafft neue Möglichkeiten der sozialen Interaktion. AR-Spiele wie Pokémon GO, Ingress und Zombies, Run! sind bekannte Beispiele für diese Technologie in Aktion. Die körperliche Bewegung, die diese Spiele erfordern, kann jedoch ein Risiko für die Nutzer darstellen, insbesondere wenn sie nicht sicher und verantwortungsvoll genutzt werden. Insgesamt hat AR das Potenzial, die Spieleindustrie zu verbessern und neue und aufregende Erlebnisse für die Spieler zu schaffen. \cite{Das2017AugmentedRV}


