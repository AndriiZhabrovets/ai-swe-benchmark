
\section{Anwendungsgebiete von AR}

\subsection{Archäologie}

Die AR-Technologie wurde in der Archäologie vor allem im Bereich des Augmented Tourism eingesetzt, um den Besuch historischer Stätten durch die Überlagerung virtueller Rekonstruktionen mit der realen Umgebung angenehmer zu gestalten. Dies wird durch Geräte wie Head-Mounted Displays (HMDs) oder mobile Geräte erreicht, die virtuelle Objekte über eine Live-Videoübertragung legen. Herausragende Beispiele sind “ARCHEOGUIDE” und das Projekt “Cultural Heritage Experiences through Socio-personal Interactions and Storytelling”. 

Über den virtuellen Tourismus hinaus kann AR auch helfen, archäologische Stätten aus einer phänomenologischen Perspektive zu erkunden. Durch die Überlagerung von geografischen Daten aus einem Geografischen Informationssystem (GIS) mit der realen Welt wird eine verkörperte Erfahrung der Landschaft ermöglicht. Dieses Konzept wird anhand eines Projekts in einer bronzezeitlichen Siedlung in Cornwall, Grossbritannien, demonstriert, bei dem eine massstabsgetreue Rekonstruktion der Rundhäuser der Siedlung mit Hilfe einer Spiele-Engine, GIS-Software, einem AR-Plugin und benutzerdefinierten Skripten über die reale Landschaft gelegt wurde. 
\cite{Archaeology}

\subsection{Übersetzung}

Die Übersetzung ist eine der am meisten genutzten Anwendungen der AR-Technologie, auch wenn einige Leute vielleicht nicht wissen, dass es eine ist. 

Das bekannteste Beispiel ist eine mobile Anwendung namens Word Lens, die später von Google gekauft und in die Google Translate-App integriert wurde. Mit dieser Anwendung kann man die Kamera eines Smartphones auf einen fremden Text richten, z. B. auf Schilder oder Menüs. Die App zeigt dann eine sofortige Übersetzung des Textes auf dem Bildschirm an, genau dort, wo der fremde Text steht. Auf diese Weise ist es möglich, den Text in Echtzeit zu verstehen, ohne dass ein Foto oder eine Eingabe in die Anwendung erforderlich ist. Aufgrund seiner Nützlichkeit wurde es zu einem praktischen Tool für Reisende und Sprachschüler auf der ganzen Welt, die verschiedene Sprachen leichter verstehen und mit ihnen interagieren möchten. \cite{QuestVisual_2010}


\subsection{Bildungsbereich}

Augmented Reality (AR) kann im Bildungsbereich eingesetzt werden, um interaktive und immersive Lernumgebungen zu schaffen, in denen Lernende virtuelle Objekte und Informationen in realen Kontexten erkunden und manipulieren können. AR kann in verschiedene Lernaktivitäten wie Simulationen, Spielen und Prüfungen integriert werden, die die Entwicklung kognitiver, affektiver und psychomotorischer Fähigkeiten fördern können. AR kann den Lernenden auch personalisierte und adaptive Lernerfahrungen bieten, die Interesse, Motivation und Zusammenarbeit verbessern können. Zu den pädagogischen Vorteilen von AR gehören eine bessere Wissensspeicherung, verbesserte Problemlösungs- und Entscheidungsfähigkeiten sowie mehr Kreativität und Innovation. Der Einsatz von AR im Bildungsbereich bringt jedoch auch einige Herausforderungen und Einschränkungen mit sich, wie z. B. die Notwendigkeit geeigneter Hard- und Software, mangelnde Standardisierung und Kompatibilität sowie potenzielle negative Auswirkungen auf die Lernergebnisse, wenn sie nicht effektiv konzipiert und umgesetzt werden. Daher müssen Pädagogen und Designer die pädagogischen, technologischen und ethischen Fragen, die mit dem Einsatz von AR in der Bildung verbunden sind, sorgfältig abwägen und geeignete Strategien und Richtlinien entwickeln, um den potenziellen Nutzen zu maximieren. \cite{Wu2013CurrentSO}

\subsection{Videospiele}

Einer der Hauptvorteile der Augmented-Reality-Technologie (AR) in der Videospielindustrie besteht darin, dass sie es den Nutzern ermöglicht, mit synthetischen Objekten, Personen und Umgebungen zu interagieren, die über reale Umgebungen gelegt werden, und ihre Erfahrungen in diesen Umgebungen mit computergenerierten Bildern und Tönen zu erweitern. Dies verbessert das Spielerlebnis und schafft neue Möglichkeiten der sozialen Interaktion. AR-Spiele wie Pokémon GO, Ingress und Zombies, Run! sind bekannte Beispiele für diese Technologie in Aktion. Die körperliche Bewegung, die diese Spiele erfordern, kann jedoch ein Risiko für die Nutzer darstellen, insbesondere wenn sie nicht sicher und verantwortungsvoll genutzt werden. Insgesamt hat AR das Potenzial, die Spieleindustrie zu verbessern und neue und aufregende Erlebnisse für die Spieler zu schaffen. \cite{Das2017AugmentedRV}

\subsection{Medizin}
Augmented Reality (AR) Technologie hat verschiedene Anwendungen im medizinischen Bereich. AR-Technologie kann eingesetzt werden, um interaktive und immersive Lernerfahrungen für Medizinstudenten zu schaffen. Sie kann dabei helfen, komplexe medizinische Konzepte und Verfahren auf ansprechendere Weise zu veranschaulichen. 

AR kann auch eingesetzt werden, um Operationen und andere medizinische Verfahren zu simulieren, so dass Studenten ihre Fähigkeiten in einer sicheren und kontrollierten Umgebung üben und verbessern können. 

Ausserdem kann es eingesetzt werden, um während einer Operation virtuelle Bilder über den Körper des Patienten zu legen und den Chirurgen in Echtzeit Anweisungen und Informationen zu geben. Dies kann dazu beitragen, die Genauigkeit zu verbessern und das Risiko von Komplikationen bei komplexen chirurgischen Eingriffen zu verringern. 

AR kann zur Erstellung interaktiver Rehabilitationsprogramme für Patienten eingesetzt werden, die sich von Verletzungen oder Operationen erholen. Sie kann den Patienten visuelle Rückmeldungen und Anleitungen geben und ihnen helfen, Übungen korrekt auszuführen und ihre Fortschritte im Laufe der Zeit zu verfolgen. 

Zusätzlich kann AR-Technologie zur Verbesserung der medizinischen Bildgebung eingesetzt werden, indem virtuelle Bilder über den Körper des Patienten gelegt werden. Dies kann dazu beitragen, die Genauigkeit von Diagnosen und Behandlungsplänen zu verbessern. 

Und schliesslich kann Augmented Reality auch für die medizinische Beratung und Unterstützung aus der Ferne eingesetzt werden. Sie ermöglicht es Ärzten, Patienten in Echtzeit zu sehen und mit ihnen zu interagieren, selbst wenn sie sich an verschiedenen Orten befinden. \cite{Bhatla2022AugmentedRT}, \cite{Parsons2021CurrentPO}, \cite{Hsieh2018PreliminarySO}

\subsection{Militär}
Die AR-Technologie wurde in verschiedenen militärischen Anwendungen eingesetzt, und es gibt mehrere reale Beispiele für ihren Einsatz im Militär. Ein Beispiel ist die militärische Ausbildung, wo die AR-Technologie an der Heeresakademie eingesetzt wurde, um Szenarien für Militärstudenten zu erstellen, die es ihnen ermöglichen, Ausbildungsszenarien für Bediener zu modellieren und zu simulieren und aktiver daran teilzunehmen. 

Ein weiteres Beispiel ist die Situationswahrnehmung, bei der AR-Technologie die Situationswahrnehmung durch die Bereitstellung von Echtzeitinformationen für militärisches Personal verbessern kann. Die AR-Technologie kann Informationen in Echtzeit liefern, die eine nicht zu vernachlässigende Anwendungsperspektive im gesamten Lebenszyklus militärischer Aktivitäten und Handlungen der Betreiber haben werden. 

AR/VR-Technologien wurden auch als kosteneffiziente Lösung zur Verbesserung der Ausbildung von Soldaten identifiziert, und AR-Technologien können eingesetzt werden, um die Ausbildung an die Bedürfnisse der Soldaten anzupassen. 

Schliesslich können mit AR-Technologien computergenerierte Bilder in die reale Welt eingeblendet werden, was bei der Planung militärischer Operationen hilfreich sein kann. 

Insgesamt haben AR-Technologien das Potenzial, die Effizienz militärischer Aktivitäten und Handlungen zu verbessern, das Situationsbewusstsein zu erhöhen und das militärische Personal in Echtzeit mit Informationen zu versorgen. \cite{Virca2021ApplicationsOA}



\subsection{Navigation}
Wie bereits im geschichtlichen Teil dieses Artikels erwähnt, verwendete das NASA-Raumschiff X-38 ein hybrides System für künstliches Sehen. Dieses System überlagerte Videobilder mit Kartendaten, um die Navigationsfähigkeiten des Raumfahrzeugs zu verbessern. Es basierte auf der Software LandForm, die sich besonders bei schlechten Sichtverhältnissen bewährte. \cite{Delgado2001HybridSS}

Darüber hinaus kann Augmented Reality (AR) zur Verbesserung von Navigationssystemen eingesetzt werden, indem virtuelle Objekte in die reale Welt eingeblendet werden, was zu einer realistischeren und benutzerfreundlicheren Erfahrung führt. AR kann den Nutzern intuitivere und realistischere Navigationsanweisungen geben, z. B. durch Hervorheben von Orientierungspunkten oder Einblenden von Pfeilen in das Sichtfeld des Nutzers. AR kann auch die Sicherheit erhöhen, da der Nutzer seine Aufmerksamkeit nicht mehr von der Strasse ablenken muss. Das Schweizer Unternehmen Way Pay hat beispielsweise holografische AR-Navigationssysteme entwickelt, bei denen holografische optische Elemente verwendet werden, um alle streckenbezogenen Informationen, einschliesslich Wegbeschreibungen, wichtige Mitteilungen und interessante Punkte, direkt in das Sichtfeld des Fahrers und weit vor das Fahrzeug zu projizieren. \cite{WayRay}


\subsection{Architektur}
Einer der Hauptvorteile des Einsatzes von AR-Technologie in der Architektur ist die Möglichkeit, intuitive Visualisierungsplattformen für die effiziente Nutzung digital verwalteter Informationen bereitzustellen. Durch die Integration von AR-Technologie in die Architektur können Architekten immersive Erfahrungen für Kunden schaffen, die es ihnen ermöglichen, Entwürfe auf realistischere Weise zu visualisieren. Dies kann dazu beitragen, die Kommunikation zwischen Architekten und Kunden zu verbessern und ein besseres Verständnis des Entwurfs vor Baubeginn zu ermöglichen. 

AR-Technologie kann auch zur Unterstützung von Bauarbeitern auf der Baustelle eingesetzt werden, da sie den Entwurf im physischen Raum sehen können. Dies kann dazu beitragen, Fehler zu reduzieren und die Effizienz zu steigern, da die Arbeiter genau sehen können, wo jedes Bauteil platziert werden muss. Darüber hinaus kann die AR-Technologie genutzt werden, um Wartungs- und Reparaturanweisungen für Gebäudesysteme bereitzustellen, was dazu beitragen kann, Ausfallzeiten zu reduzieren und die Sicherheit zu erhöhen. 

Ein weiterer Vorteil des Einsatzes von AR-Technologie in der Architektur ist die Möglichkeit, über ubiquitäre Dienste auf Informationen vor Ort zuzugreifen. Dies bedeutet, dass Architekten und Bauarbeiter jederzeit und überall auf Informationen über das Projekt zugreifen können. Dies kann dazu beitragen, die Zusammenarbeit zwischen den Teammitgliedern zu verbessern und einen effizienteren Arbeitsablauf zu ermöglichen. 

Insgesamt kann der Einsatz von AR-Technologie in der Architektur zu einer höheren Produktivität, Sicherheit und Effizienz führen. Durch die Nutzung der neuesten Fortschritte in der AR-Technologie können Architekten und Bauarbeiter bessere Entwürfe erstellen, Fehler reduzieren und die Kommunikation und Zusammenarbeit verbessern. \cite{Chi2013ResearchTA}

\subsection{Handel und Marketing}

Die Technologie der erweiterten Realität (Augmented Reality, AR) wird zunehmend in Handel und Marketing eingesetzt, um das Kundenerlebnis und die Kundenbindung zu verbessern. Die AR-Technologie ermöglicht es den Kunden, mit den Produkten auf eine immersivere und personalisiertere Weise zu interagieren, was zu einer höheren Markentreue und höheren Umsätzen führt.

Eine mögliche Anwendung der AR-Technologie in Handel und Marketing ist die virtuelle Anprobe. Mit Hilfe der AR-Technologie kann eine virtuelle Umkleidekabine geschaffen werden, in der Kunden Kleidung, Accessoires und Make-up anprobieren können, ohne physisch im Geschäft anwesend zu sein. Diese Technologie wird von Unternehmen wie IKEA eingesetzt, um den Kunden die Möglichkeit zu geben, vor dem Kauf zu sehen, wie die Produkte bei ihnen zu Hause aussehen würden. \cite{IKEA_2017}

Eine weitere Anwendung der AR-Technologie ist die Produktvisualisierung. Mit Hilfe der AR-Technologie können 3D-Modelle von Produkten erstellt werden, mit denen Kunden in einem virtuellen Raum interagieren können. Diese Technologie wird von Unternehmen wie BMW und Converse eingesetzt, um Kunden die Möglichkeit zu geben, zu sehen, wie Produkte in der Realität aussehen und funktionieren würden. \cite{Boeriu_2022}, \cite{Online_Campaigns_2012}

AR-Technologie kann auch für standortbezogenes Marketing eingesetzt werden. Unternehmen können die AR-Technologie nutzen, um interaktive Erlebnisse für Kunden an bestimmten Orten zu schaffen. Beispielsweise könnte ein Unternehmen eine AR-Schnitzeljagd in einem Einkaufszentrum veranstalten oder die AR-Technologie nutzen, um Informationen über Sehenswürdigkeiten in einer Stadt bereitzustellen.

Zusammenfassend lässt sich sagen, dass die AR-Technologie das Potenzial hat, die Art und Weise, wie Handel und Marketing betrieben werden, zu revolutionieren. Indem sie ihren Kunden immersive und personalisierte Erlebnisse bieten, können Unternehmen das Engagement, die Markentreue und den Umsatz steigern.



