\section{Introduction}

The rapid advancements in Artificial Intelligence (AI) have brought transformative changes to numerous industries, with software engineering being a significant focus. AI-driven systems have moved beyond automation of routine tasks and now exhibit the potential to tackle complex challenges, including programming problem-solving. These developments have prompted critical discussions about whether AI can assume roles traditionally held by software engineers. Understanding the capabilities and limitations of these systems is essential for gauging their potential impact on the software development process.

This study explores the problem-solving capabilities of state-of-the-art AI models by evaluating their performance on programming tasks. Using a rigorous benchmarking framework, we assess these models through a set of common problems sourced from Leetcode, a platform widely acknowledged for its relevance in evaluating programming proficiency. By providing consistent prompts and analyzing their outputs, the research seeks to measure the effectiveness of AI in solving tasks of varying complexity and to draw meaningful comparisons with human performance.

The investigation centers on analyzing the technical skills of AI in problem-solving, identifying strengths and weaknesses, and determining its readiness to operate independently or alongside human developers. In doing so, this study also examines broader questions about the evolving role of AI in software engineering and its implications for the future of the profession.

The paper is structured to provide a comprehensive analysis of the topic, beginning with a review of existing literature to situate the research within the broader context of AI applications in software development. A detailed methodology follows, outlining the design and implementation of the benchmarks used for evaluating AI performance. The results are presented and analyzed to highlight key patterns and insights, culminating in a discussion of their implications for software engineering and AI research. Finally, the study concludes by reflecting on the findings and their relevance to the question of whether AI is capable of substituting software engineers.

By addressing these themes, this research aims to contribute to a deeper understanding of the capabilities of AI in software development and to inform future directions for improving AI systems in this domain.
