\section{Introduction}

The rapid evolution of artificial intelligence (AI) has brought significant attention to its transformative potential in numerous fields, particularly in software engineering. Programming, as one of the core aspects of software development, stands out due to the increasing capabilities of AI to understand, generate, and debug code. These advancements open doors to streamlining complex development processes, assisting engineers, and even automating tasks that traditionally required human intervention. However, these innovations bring with them a host of challenges, ranging from concerns about reliability to questions of adaptability and efficiency in real-world applications.

This study emerges from the pressing need to evaluate AI-driven solutions systematically. It aims to understand whether and how AI can substitute or complement human programmers. Through a structured framework, this research seeks to provide detailed insights into the performance of AI in programming tasks, ensuring that the findings are grounded in rigorous, real-world testing conditions. By doing so, it contributes to the broader discussion on the future role of AI in the software industry.

\subsection{Motivation}
The motivation behind this research stems from the growing reliance on AI across various sectors and the unique challenges this brings to software engineering. As AI continues to excel in automating repetitive tasks and solving structured problems, its potential to redefine programming roles becomes more apparent. The ability of AI to autonomously generate code, optimize algorithms, and debug complex systems has sparked excitement and skepticism alike. This duality motivates the need to assess AI's capabilities systematically, understanding its limitations and strengths.

Furthermore, the increasing prevalence of AI-assisted tools such as GitHub Copilot and ChatGPT has already begun transforming workflows, reducing the time and effort needed for specific tasks. This shift raises critical questions: Can AI match or surpass human expertise in solving programming challenges? Can it be entrusted with mission-critical responsibilities? These questions form the cornerstone of this research, underscoring the urgency to develop benchmarks that measure AI's real-world applicability and reliability.

\subsection{Research Objectives}
This study proposes a benchmark specifically designed to evaluate AI models' proficiency in handling programming tasks, with an emphasis on solving algorithmic challenges. The benchmark incorporates a curated collection of well-known problems that mimic real-world scenarios encountered by human developers. By using standardized metrics such as execution time, memory usage, and accuracy, the study aims to capture the technical capabilities of AI in coding. While the primary focus is on assessing the models’ ability to solve isolated algorithmic problems, the study intentionally excludes broader considerations like code readability, maintainability, or integration into complex software systems. These dimensions require different evaluation methods and are beyond the scope of this work. By concentrating on algorithmic performance, the benchmark offers a clear and focused view of current AI capabilities. This structured approach provides a solid foundation for future research, paving the way for more comprehensive evaluations and applications.
\subsection{Scope of the Research}
This paper provides an in-depth exploration of the research, starting with the motivations for assessing AI's programming potential. The introduction outlines the benchmark's scope, objectives, and areas of focus, establishing the groundwork for the study. Following this, the next section delves into the background and related work, offering context on AI in software engineering and the role of benchmarks in evaluating AI capabilities. The paper then transitions to a detailed discussion of the benchmark’s design and implementation, encompassing problem selection, technological tools, and evaluation metrics. Results and analysis follow, highlighting key findings, identifying patterns, and discussing unexpected insights. A dedicated section addresses challenges encountered during the research process and the adaptations made to overcome them. The paper concludes with a discussion of the broader implications of the findings, limitations of the benchmark, and its potential for future research. Finally, forward-looking recommendations suggest strategies for expanding and improving the benchmark, including exploring additional AI models and refining evaluation methodologies.