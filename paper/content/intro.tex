\section{Introduction}

Diese Arbeit befasst sich mit der Frage, was Augmented-Reality-Technologie (“Erweiterte Realität”, AR) ist und wie sie in der realen Welt eingesetzt wird. Zunächst werden mehrere Definitionen des Konzepts der Augmented Reality betrachtet, darunter die Definitionen der Wissenschaftler Milgram, Takemura, Utsumi und Kishino aus dem Jahr 1994 und eine spätere Definition von T. Azuma aus dem Jahr 1997. Anschliessend werden die beiden Hauptkomponenten des Augmented-Reality-Systems, Hardware und Software, erörtert. Um die Entwicklung der AR-Technologie besser zu verstehen und einen Einblick in die Nutzung von AR im Laufe ihrer Existenz zu erhalten, werden wichtige Ereignisse, die die Technologie beeinflusst haben, auch chronologisch betrachtet, und zwar von ihrer Erfindung im Jahr 1968 bis zur Einführung von “Apple Vision Pro” im Jahr 2023, einem hochmodernen AR-Gerät, das von dem bekannten amerikanischen Technologieunternehmen Apple entwickelt wurde. \cite{Apple}   

Im Anschluss werden unterschiedliche Anwendungen der AR wie in der Archäologie, Bildung, Medizin, Militär, in Videospiele, Navigation und Übersetzung besprochen. In der Arbeit wird untersucht, wie die AR-Technologie dazu beigetragen hat, die Erfahrungen zu verbessern, die Effizienz zu steigern und die Möglichkeiten in diesen Bereichen zu erweitern. Durch einen umfassenden Überblick über aktuelle AR-Implementierungen werden die konkreten Vorteile erläutert, die bereits verwirklicht wurden. Die Rolle der Augmented Reality (AR) in der Archäologie besteht zum Beispiel darin, digitale Rekonstruktionen antiker Zivilisationen zu ermöglichen. Dies zeigt das immense Potenzial von AR, um die historische Forschung und Bewahrung zu revolutionieren. In der Bildung wird AR genutzt, um Studenten in immersive Lernerfahrungen einzubeziehen und so ein tieferes Verständnis von komplexen Themen zu fördern sowie das Behalten des Gelernten zu erleichtern. Auch Mediziner profitieren von AR, indem es ihnen beim Einsatz in der chirurgischen Navigation hilft, präzise Eingriffe zu erleichtern und Risiken zu minimieren.

Unter Berücksichtigung der oben genannten Beispiele, wie die AR-Technologie viele Lebensbereiche verbessern kann, bietet diese Arbeit einen Ausgangspunkt für jeden, der sich für die Erforschung oder Nutzung von Augmented Reality interessiert.

% Obwohl diese Studie die Herausforderungen und Grenzen der AR-Technologie einräumt, kommt sie letztendlich zu dem Schluss, dass die Vielzahl der möglichen Anwendungen und die damit verbundenen Vorteile bei weitem überwiegen. Die Fähigkeit von AR, unser Leben zu bereichern - angefangen von der Bewahrung unseres kulturellen Erbes bis hin zur Verbesserung des Lernens, der Gesundheitsversorgung und der Neudefinition von Unterhaltung - zeigt ihr immenses Potenzial. Dieses Papier fordert dazu auf, die AR Technologie weiter zu erforschen und zu innovieren. Es ermutigt Forscher, Industrie und politische Entscheidungsträger dazu, diese dynamische Technologie im Sinne des Gemeinwohls zu nutzen. Mit der fortschreitenden Entwicklung der AR bietet sie eine transformative Wirkung auf verschiedene Bereiche und hat das Potenzial, die Art und Weise, wie wir mit der Welt interagieren und sie wahrnehmen, neu zu gestalten.


