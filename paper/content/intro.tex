\section{Introduction}

\subsection{Motivation}
The rapid growth of artificial intelligence (AI) has sparked widespread enthusiasm about its potential to revolutionize software engineering. Among the many areas poised for transformation, programming stands out as particularly exciting. AI’s capability to understand, generate, and debug code presents opportunities to streamline development processes, assist engineers, and potentially take over certain tasks traditionally performed by humans. However, these advancements also prompt important questions about the reliability, efficiency, and adaptability of AI-driven solutions. To address these questions, it is crucial to develop rigorous methods of evaluation. This research is motivated by the need to systematically assess the real-world programming abilities of current AI models, providing a structured framework to gauge their performance.
\subsection{Research Objectives}
This study proposes a benchmark specifically designed to evaluate AI models' proficiency in handling programming tasks, with an emphasis on solving algorithmic challenges. The benchmark incorporates a curated collection of well-known problems that mimic real-world scenarios encountered by human developers. By using standardized metrics such as execution time, memory usage, and accuracy, the study aims to capture the technical capabilities of AI in coding. While the primary focus is on assessing the models’ ability to solve isolated algorithmic problems, the study intentionally excludes broader considerations like code readability, maintainability, or integration into complex software systems. These dimensions require different evaluation methods and are beyond the scope of this work. By concentrating on algorithmic performance, the benchmark offers a clear and focused view of current AI capabilities. This structured approach provides a solid foundation for future research, paving the way for more comprehensive evaluations and applications.
\subsection{Scope of the Research}
This paper provides an in-depth exploration of the research, starting with the motivations for assessing AI's programming potential. The introduction outlines the benchmark's scope, objectives, and areas of focus, establishing the groundwork for the study. Following this, the next section delves into the background and related work, offering context on AI in software engineering and the role of benchmarks in evaluating AI capabilities. The paper then transitions to a detailed discussion of the benchmark’s design and implementation, encompassing problem selection, technological tools, and evaluation metrics. Results and analysis follow, highlighting key findings, identifying patterns, and discussing unexpected insights. A dedicated section addresses challenges encountered during the research process and the adaptations made to overcome them. The paper concludes with a discussion of the broader implications of the findings, limitations of the benchmark, and its potential for future research. Finally, forward-looking recommendations suggest strategies for expanding and improving the benchmark, including exploring additional AI models and refining evaluation methodologies.