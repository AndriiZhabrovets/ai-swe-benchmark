% --------------------------------------------------------------------------------------------------------
%%% SETUP SECTION
% --------------------------------------------------------------------------------------------------------


% First we declare the type of the document. 
% Depending on the type some factors may change.
% Here we also enter the size of the paper and many other parameters
\documentclass[a4paper,12pt,twoside]{article}

% a4paper – A4 paper size
% 12pt – default font size
% twoside – twoside paper format

% --------------------------------------------------------------------------------------------------------
%% IMPORT SECTION

% In LaTex we can also import packages like in any programming language
% Let's go through the most important ones.

\usepackage{graphicx}
\usepackage[german]{babel}
\usepackage[fixlanguage]{babelbib}
\usepackage[procnames]{listings}
\usepackage{xcolor}
\usepackage{geometry}
\geometry{
    a4paper, 
    left=25mm,
	top=25mm,
	textwidth=16cm,
	textheight=23cm
}
\usepackage{siunitx} 
\usepackage{amsmath,amssymb}
\usepackage{datetime} 
\usepackage{color}
\usepackage{float}
\usepackage{mathptmx}
\usepackage{enumitem}
\usepackage[T1]{fontenc}
\usepackage[utf8]{inputenc}
\usepackage{lipsum}
\usepackage{setspace}
\setstretch{1.5}
\usepackage{hyperref} 
\usepackage{apacite}
\bibliographystyle{apacite}
\usepackage{circuitikz}
\usepackage{tikz}
\usetikzlibrary{arrows}
\usepackage{adjustbox}
% graphicx - adds additional graphical features to LaTeX
% babel - allows you to use special symbols from other languages
% listing – adds code spaces to LaTeX.
% xcolor – allows to work with colors in LaTeX. Works well with previous package
% geometry – a beginner-friendly package that allow you to work in a simple way with layout of everything (margins, headers, footnotes)
    % textwidth and texthight mean the size of the textarea
% siunix – for writing numbers with units 
% amsmath –  better mathematical formulas
% amssymb –  more math symbols
% datetime – date and time formatting
% color – a more basic version of xcolor.
% float – without it, tables position more or less random, with this an writing \begin{table}[H] the table appears right where it is defined!
% mathptmx – a package for adding Times Roman font in both text and math formulas
% enumitem – extensive customization of lists, including modifying labels, adjusting spacing, and changing the appearance of list items.
% fontenc – allows to select font encoding
    % T1 – refers to a 8-bit font encoding, more complex compared to a 7-bit one.
% inputenc - Unicode characters directly in your LaTeX source code. It is necessary when you need to use characters outside of the ASCII range. By specifying the input encoding as the one you enter in parameters, you ensure that all displayable characters in selected encoding are available for input.
    % utf8 - defines encoding
% lipsum – easy and quick way for creating lorem ipsum (dummy) text
% setspace - control the line spacing in your LaTeX document. It provides options and user-level commands to change the spacing between lines of text. The package is commonly used when you want to modify the line spacing for a specific section or for the entire document.
    % with setstretch we change the line spacing.
% hyperref - creating hyperlinks in LaTeX documents. Can be used for bookmarks and citations.
% apacite - formatting citations and references in APA (American Psychological Association) style.
    % The line \bibliographystyle{apacite} specifies the style to be used for the bibliography.
% circuitikz – creating circuit diagrams
% tikz – a powerful tool for creating schemes and diagrams in LaTeX.


%% END OF IMPORT SECTION 
% --------------------------------------------------------------------------------------------------------
%% ADDITIONAL INPUT

% We use input command when we want to import the content of another .tex file into the current one. So it is again very similar to the importing of local files in python.
% At this part it is also good to "import" some extra formatting and define your custom commands. For convenience and cleanliness of our code we are going to do it in separate files.

\input{utilities/commands.tex}
\input{utilities/code_formatting.tex}

%% END OF ADDITIONAL INPUT
% --------------------------------------------------------------------------------------------------------
%%% END OF SETUP SECTION 
% --------------------------------------------------------------------------------------------------------


% --------------------------------------------------------------------------------------------------------
%%% ACTUAL DOCUMENT 
% --------------------------------------------------------------------------------------------------------


% The next part is the body of the document
\begin{document}
    % Before we start adding anything it would be a good idea to define the language of the document

    \selectlanguage{german} % or german

    % Now we input our title page
    \begin{titlepage}
    % First we create an empty page
    \clearpage\thispagestyle{empty}
    \setstretch{1}

    % So here we are creating a kind of a header section (align to the top of the page ) which consists of two another sections which both take 50% (0,5) of the mother section: 
    \begin{minipage}[t]{\textwidth}
        \begin{minipage}[t]{0.5\textwidth}
            Andrii Zhabrovets\\
            Friedrichshafnerstrasse 54a\\
            8590 Romanshorn\\
            076 525 13 74\\
            anzhabro@ksr.ch
        \end{minipage}
        \begin{minipage}[t]{0.5\textwidth}
            \begin{flushright}
                Kantonsschule Romanshorn\\
                Klasse 3Mez\\
                SLA
            \end{flushright}
        \end{minipage}
    \end{minipage}

    % Now we make some vertical space
    \vspace{4cm}


    % We separate the next part of the code so that the title formatting applies only to the selected section
    {
    % Center our title in the middle
    \centering

    % Add title (huge size, bold) and \par is just to start new paragraph
    \Huge\bfseries Was ist die AR-Technologie (Augmented Reality) und welche Anwendungen gibt es in der realen Welt?\par

    \vspace{1cm}

    \includegraphics[width=0.5\textwidth]{attachments/ar.png}\par
    }

    % Now let's add a footer.
    \vspace{46mm}

    % No indent is similar to strip() in python
    \noindent
    Fach: Informatik\noindent \\ 
    Betreuungsperson: Dr. Andreas Schärer \\
    Abgabetermin: 11 September 2023 \\

\end{titlepage}


    % Page Numbering type. For abstract and table of contents we use roman numbers.
    \pagenumbering{roman}

    
\section*{Abstract}

% Die Technologie der Augmented Reality (Erweiterte Realität, AR), die digitale und physische Welten nahtlos miteinander verbindet, hat in den letzten Jahren zunehmend an Bedeutung gewonnen. Diese Arbeit untersucht die AR-Technologie aus theoretischer und technologischer Sicht und zeichnet ihre Entwicklung seit ihren Anfängen nach.  Die Arbeit untersucht gründlich die Vor- und Nachteile von aktuellen und möglichen Anwendungen von Augmented Reality in verschiedenen Bereichen wie Archäologie, Bildung, Medizin, Militär, Videospielen, Navigation und Übersetzung.
\textbf{To Be Done}



    \newpage
    \tableofcontents

    \parindent=0pt
    \parskip=6pt

    \newpage

    \pagenumbering{arabic}

    \section{Introduction}

\subsection{Motivation}
The rapid growth of artificial intelligence (AI) has sparked widespread enthusiasm about its potential to revolutionize software engineering. Among the many areas poised for transformation, programming stands out as particularly exciting. AI’s capability to understand, generate, and debug code presents opportunities to streamline development processes, assist engineers, and potentially take over certain tasks traditionally performed by humans. However, these advancements also prompt important questions about the reliability, efficiency, and adaptability of AI-driven solutions. To address these questions, it is crucial to develop rigorous methods of evaluation. This research is motivated by the need to systematically assess the real-world programming abilities of current AI models, providing a structured framework to gauge their performance.
\subsection{Research Objectives}
This study proposes a benchmark specifically designed to evaluate AI models' proficiency in handling programming tasks, with an emphasis on solving algorithmic challenges. The benchmark incorporates a curated collection of well-known problems that mimic real-world scenarios encountered by human developers. By using standardized metrics such as execution time, memory usage, and accuracy, the study aims to capture the technical capabilities of AI in coding. While the primary focus is on assessing the models’ ability to solve isolated algorithmic problems, the study intentionally excludes broader considerations like code readability, maintainability, or integration into complex software systems. These dimensions require different evaluation methods and are beyond the scope of this work. By concentrating on algorithmic performance, the benchmark offers a clear and focused view of current AI capabilities. This structured approach provides a solid foundation for future research, paving the way for more comprehensive evaluations and applications.
\subsection{Scope of the Research}
This paper provides an in-depth exploration of the research, starting with the motivations for assessing AI's programming potential. The introduction outlines the benchmark's scope, objectives, and areas of focus, establishing the groundwork for the study. Following this, the next section delves into the background and related work, offering context on AI in software engineering and the role of benchmarks in evaluating AI capabilities. The paper then transitions to a detailed discussion of the benchmark’s design and implementation, encompassing problem selection, technological tools, and evaluation metrics. Results and analysis follow, highlighting key findings, identifying patterns, and discussing unexpected insights. A dedicated section addresses challenges encountered during the research process and the adaptations made to overcome them. The paper concludes with a discussion of the broader implications of the findings, limitations of the benchmark, and its potential for future research. Finally, forward-looking recommendations suggest strategies for expanding and improving the benchmark, including exploring additional AI models and refining evaluation methodologies.

    \newpage


    \section{Augmented Reality}


\subsection {Definition}

Was ist AR?
Das Konzept der erweiterten Realität (Augmented Reality, AR) wurde von vielen Forschern der Informatik unter Berücksichtigung verschiedener Faktoren unterschiedlich definiert.

Die ersten Definitionen wurden von Milgram, Takemura, Utsumi und Kishino vorgeschlagen
(1994). In ihrer Arbeit “Augmented Reality: A class of displays on the reality-virtuality continuum” (Eine Klasse von Displays auf dem Realitäts-Virtualitäts-Kontinuum) konzeptualisierten sie das Virtual-Reality-Kontinuum, das vier Systeme berücksichtigt: reale Umgebung (Real Environment), erweiterte Realität (Augmented Reality), erweiterte Virtualität (Augmented Virtuality) und virtuelle Umgebung (Virtual Environment). Darüber hinaus wurden zwei Ansätze zur Definition von “Augmented Reality” genannt: eine breite Definition und eine präzisere Definition. In der breiten Definition wird Augmented Reality definiert als “Erweiterung des natürlichen Feedbacks an den Operator mit simulierten Hinweisen”. Im Gegensatz dazu wird sie im genaueren Verständnis definiert als “eine Form der virtuellen Realität, bei der das am Kopf montierte Display des Teilnehmers transparent ist und eine klare Sicht auf die reale Welt ermöglicht”.  \cite{Milgram94a} \cite{Milgram94b}
\vspace{1cm}
\begin{figure}[h!]
    \centering
    \includegraphics[width=0.9\textwidth]{attachments/MR_Definition.png}
    \caption{Vereinfachte Darstellung eines “Reality-Virtuality (RV) Continuum”.}  \cite{Milgram94a}
    \end{figure}
    

    \newpage

Andere Forscher, wie z. B. T. Azuma, definieren AR anhand seiner Merkmale oder Eigenschaften. Nach seinem Ansatz wird AR als Systeme definiert, die die folgenden drei Merkmale aufweisen:
\begin{enumerate}
    \item kombiniert reale und virtuelle Objekte in einer realen Umgebung;
    \item läuft interaktiv und in Echtzeit;
    \item registriert (richtet aus) in 3-D reale und virtuelle Objekte miteinander.  \cite{Azuma1997ASO}
\end{enumerate}


\subsection {Schlüsseltechnologien}

% An Augmented Reality System can be divided into two components: one is hardware, and the other is software  \cite{craig2013understanding}, \cite{Chatzopoulos}.
Ein Augmented-Reality-System kann in zwei Komponenten unterteilt werden: die Hardware und die Software \cite{craig2013understanding}, \cite{Chatzopoulos}.

% The main characteristic of the hardware components is to acquire and display the data and information, and process it. The \textbf{input components} are sensors that respond to physical or chemical stimuli from the real environment and provide the necessary data for the development of the system. The \textbf{output components} are devices for displaying the information, which can be divided into wearable and non-wearable, as well as optical, video, and projection devices. The output itself can be further divided into two variants of displays: \textbf{Optical See-through Display}, where the virtual contents are projected onto the interface to mix with the real scene optically, and \textbf{Video See-through Display}, which has two work modalities: one uses HMD (Head-Mounted Display) devices, and the other works with camera and screen in handheld devices, such as smartphones and tablets \cite{Azuma1997ASO}. 
Das Hauptmerkmal der Hardwarekomponenten ist die Erfassung und Anzeige der Daten und Informationen sowie deren Verarbeitung. Die \textbf{Input-Komponenten} sind Sensoren, die auf physikalische oder chemische Reize aus der realen Umgebung reagieren und die notwendigen Daten für die Entwicklung des Systems liefern. Die \textbf{Output-Komponenten} sind Geräte zur Anzeige der Informationen, die in tragbare und nicht tragbare sowie optische, Video- und Projektionsgeräte unterteilt werden können. Die Ausgabe selbst kann weiter in zwei Varianten von Displays unterteilt werden: \textbf{Optisch durchsichtiges Display (Optical See-through Display)}, bei dem die virtuellen Inhalte auf die Schnittstelle projiziert werden, um sich optisch mit der realen Szene zu vermischen, und \textbf{Video-Durchsichtiges Display (Video See-through Display)}, bei dem es zwei Arbeitsmodalitäten gibt: eine verwendet HMD-Geräte (Head-Mounted Display), die andere arbeitet mit Kamera und Bildschirm in Handheld-Geräten wie Smartphones und Tablets \cite{Azuma1997ASO}.

\newpage

\vspace{1cm}

\begin{figure}[h!]
    \centering
    \includegraphics[width=0.65\textwidth]{attachments/Diagram_2.png}
    \caption{Optisch durchsichtiges Display}  \cite{Azuma1997ASO}
    \end{figure}

% \textbf{Video-Durchsichtiges Display (Video See-through Display):} hat zwei Arbeitsmodalitäten: Die eine verwendet HMD(Head-Mounted Display)-Geräte, die andere arbeitet mit Kamera und Bildschirm in Handheld-Geräten. Zum Beispiel Smartphones und Tablets.

\vspace{1cm}

\begin{figure}[h!]
    \centering
    \includegraphics[width=0.75\textwidth]{attachments/Diagram_1.png}
    \caption{Video-Durchsichtiges Display}  \cite{Azuma1997ASO}
    \end{figure}

% On the software side, one of the key tasks is to derive real-world coordinates that are independent of the camera and camera images. This process is known as image registration and involves various computer vision methods, particularly those related to video tracking  \cite{Azuma2001RecentAI}.
Auf der Software-Seite besteht eine der Hauptaufgaben darin, reale Koordinaten abzuleiten, die von der Kamera und den Kamerabildern unabhängig sind. Dieser Prozess wird als Bildregistrierung bezeichnet und umfasst verschiedene Computer-Vision-Methoden, insbesondere solche, die sich auf die Videoverfolgung beziehen \cite{Azuma2001RecentAI}.

    \newpage


    
\section{Research Methodology}

\subsection{Description of the Benchmark Framework}

\subsubsection{Selection and Categorization of Programming Problems}

Die Entwicklung der Augmented-Reality-Technologie geht auf die 1960er Jahre zurück, als der Informatiker Ivan Sutherland das erste kopfgetragene Display erfand. Dieses bahnbrechende Gerät, das auch als “Sword of Damocles” bezeichnet wurde, ebnete den Weg für tragbare Computerschnittstellen, obwohl es keine echten AR-Funktionen bot \cite{Sutherland1968AHT}. Nicht lange danach, 1975, gründete Myron Krueger den Videoplace, ein Labor für künstliche Realität. Dieser Raum reagierte auf die Bewegungen und Aktionen der Benutzer und machte Brillen oder Handschuhe überflüssig \cite{Videoplace}.

\subsubsection{Characteristics of the Problem Set}

Tatsächlich wurde der Begriff “Augmented Reality” erst 1990 von Thomas P. Caudell, einem ehemaligen Forscher bei Boeing, eingeführt \cite{Lee2012AugmentedRI}. Im Jahr 1992 stellte Louis Rosenburg vom Armstrong Research Lab der United States Air Force “Virtual Fixtures” vor, eines der ersten voll funktionsfähigen Augmented-Reality-Systeme \cite{rosenberg1992use}.

\subsection{Overview of AI Models Evaluated}

Weitere Entwicklungen in den späten 90er Jahren waren die Definition des Realitäts-Virtualitäts-Kontinuums von Paul Milgram und Fumio Kishino im Jahr 1994, das sich von der realen Umgebung zur virtuellen Umgebung erstreckt. Augmented Reality (AR) und Virtual Reality (VR) liegen dazwischen, wobei sich AR zur realen Welt und VR zur virtuellen Welt neigt  \cite{Milgram94a}. Ronald Azumas Studie von 1997 über AR ist ein weiterer wichtiger Meilenstein, der eine allgemein akzeptierte Definition der Technologie liefert. Azuma definierte AR als die Integration von realen und virtuellen Umgebungen mit 3D-Registrierung und Echtzeit-Interaktivität \cite{Azuma1997ASO}. Im Jahr 2000 brachte Hirokazu Kato das AR ToolKit auf den Markt, eine Open-Source-Softwarebibliothek, die die Entwicklung von AR-Softwareanwendungen durch eine Video-Tracking-Technologie revolutionierte, die virtuelle Grafiken über die physische Welt legt \cite{ARTooLKIT}.

\subsection{Evaluation Metrics and Criteria}

In den frühen 2010er Jahren begannen grosse Technologieunternehmen in die AR-Branche einzusteigen. So stellte Google 2012 seine Google Glass vor, eine Augmented-Reality-Brille, die den Nutzern immersive Erfahrungen bietet \cite{Google_for_Developer_2012}. Im Jahr 2015 kündigte Microsoft Windows Holographic und das HoloLens-Headset für Augmented Reality an. Das Headset verschmolz hochauflösende “Hologramme” mit der physischen Welt \cite{Tech_Discussion_2015}. Im Jahr 2016 wurde die Technologie leichter zugänglich, als Niantic Pokémon Go für iOS und Android veröffentlichte. Die mobile AR-App erfreute sich grosser Beliebtheit und steigerte die Attraktivität von AR-Spielen \cite{Bond_2016}. Darüber hinaus hatte der Aufstieg mobiler AR-Apps einen bemerkenswerten Einfluss auf andere Branchen wie den Einzelhandel. Im Jahr 2017 brachte IKEA seine Augmented-Reality-App IKEA Place auf den Markt, die es den Nutzern ermöglicht, sich vor dem Kauf von Möbeln zu Hause eine Vorschau anzusehen \cite{IKEA_2017}. Im Jahr 2023 schliesslich kündigte Apple das Apple Vision Pro an, ein Augmented-Reality-Headset, das die reale und die digitale Welt “nahtlos” miteinander verschmelzen lässt und den neuesten Fortschritt in der AR-Technologie markiert \cite{Apple}.





    \newpage

    \section{Benchmark Design and Implementation}

\subsection{Problem Selection and Categorization}

During the selection process of the problems which would then be used for assessing the capabilities of AI models many aspects could be put at the priority: difficulty, variety, popularity. Taking int account the main goal of this research, the selection of problems should be based solely on how often they appear on the interviews. On the LeetCode platform, problems were filtered by the category Top Interview Questions, focusing on tasks considered crucial for professional software development. From this refined selection, the five most popular problems from each difficulty level—easy, medium, and hard—were chosen. This method ensured that the benchmark incorporated tasks ranging from fundamental algorithms to complex scenarios, as both the usage of only difficult or only simple problems would not represent fully abilities of the models.

Furthermore, choosing well-known and frequently attempted problems enhances the benchmark’s credibility and relevance. These tasks have been extensively solved and discussed by the programming community, providing the most refined solutions and this way. This factor eliminates any chance of low-key errors, which therefore makes such solutions perfect for using them as a reference. Emphasizing popularity and a balance of difficulty levels, the benchmark achieves both accessibility for AI models and a robust evaluation of their programming capabilities.

\subsection{Technologies Used}


A variety of technologies were used to effectively design and implement the benchmark. The programming language Python was chosen for the development of this benchmark because of its wide range of modules for working with data and its widespread use in AI development.

For each, the OpenAI API was used to integrate models such as GPT-3.5 and GPT-4, with plans to include Anthropic's Claude and Google's Gemini in future updates.
In fact, the availability of all the required APIs played a crucial role in the choice of programming language.

Of all the Python modules used in this project, one called "pandas" was used most often to organise, process and summarise the benchmark results, which were then exported to Excel for easy review and further visual processing. 

Visualisations of performance metrics were created using the matplotlib module. 

Execution of both AI-generated and LeetCode-parsed solutions was handled using the subprocess for running isolated blocks of code

Two built-in modules were used for monitoring, tracemalloc for memory tracking, time for runtime measurement.


\subsection{AI Models Evaluated}


The benchmark evaluated some of the leading AI models in the field, including GPT-3.5 and GPT-4 from OpenAI, Claude from Anthropic, and Gemini from Google. These models were chosen based on their popularity, accessibility, and proven capabilities in code generation and software engineering tasks.

GPT-3.5 and GPT-4: These OpenAI models represent state-of-the-art advancements in natural language understanding and generation.

Claude: Developed by Anthropic, Claude emphasizes safety and a nuanced understanding of prompts, making it a strong contender in programming tasks.

Gemini: Created by Google, Gemini is designed to tackle complex problem-solving tasks with an emphasis on general intelligence.

The inclusion of these models ensures that the benchmark provides a well-rounded and comprehensive comparison of the latest AI systems in a standardized testing environment.


\subsection{Benchmark Workflow}

The benchmark workflow was carefully structured to evaluate the performance of AI and human solutions on selected programming tasks. The process begins with loading problems from the LeetCode dataset and corresponding test cases stored in JSON files. AI solutions are generated using pre-defined prompts, while human solutions are sourced from existing implementations.

Each generated solution is executed against predefined test cases using subprocess, ensuring isolation and consistent execution. Memory usage is tracked with tracemalloc, and metrics such as runtime duration and error outputs are recorded to assess the correctness and performance of each solution. Aggregated metrics are then analyzed and visualized to facilitate clear comparisons between AI and human performance.

This structured and repeatable workflow guarantees consistency and transparency, enabling fair comparisons between different AI models and human baselines.




\subsection{Evaluation Metrics}

When evaluating the performance of AI models in solving programming problems, a set of standardized metrics is essential to ensure objectivity and consistency.
The benchmark uses a variety of metrics to thoroughly evaluate the performance of both AI and human-generated solutions. These metrics ensure a well-rounded assessment of programming capabilities.

\subsubsection{Success Rate}

This metric determines how many test cases each solution successfully solves, expressed as a percentage. It serves as the primary measure of correctness and overall effectiveness.

\subsubsection{Runtime Performance} 

This measures the time it takes for each solution to execute, expressed in milliseconds. Faster runtimes indicate better computational efficiency, which is crucial in performance-critical scenarios.

\subsubsection{Memory Usage} 

Peak memory consumption during the execution of each solution is monitored in bytes using tracemalloc. This metric assesses the resource efficiency of the implementations, particularly important for systems with limited memory resources.





% \subsection{Second Benchmark Version}









% \section{Results and Analysis}

% \subsection{Quantitative Results}

% Performance metrics (e.g., success rates on easy, medium, and hard problems).

% Comparison of models based on metrics such as speed, efficiency, and accuracy.

% Visual representation of results (tables and graphs showing success rates by difficulty).

% \subsection{Qualitative Assessment}

% Analysis of how the models approach problem-solving (e.g., whether they use brute force, optimal solutions, or heuristics).

% Examples of well-solved problems and those where AI models failed.

% Discussion of common errors (e.g., logical issues, misunderstanding problem statements, or incomplete outputs).

% \subsection{Comparison to Human Performance}

% How AI models perform in comparison to average Leetcode users (use most common solution).

% Cases where AI outperformed humans (e.g., faster execution of simple problems).

% Scenarios where humans excelled.
% \subsection{Statistical Analysis}

% Maybe some scientific methods to analyze the data.








    % Conclusion
    \newpage

    \section{Schlussfolgerung}

Augmented Reality (“Erweiterte Realität”, AR) hat in den letzten Jahren die Aufmerksamkeit von Forschern, Experten und grossen Technologieunternehmen auf sich gezogen. Diese innovative Technologie überlagert Informationen mit der physischen Welt, was zu interaktiven und immersiven Erlebnissen führt. Die in dieser Arbeit durchgeführten Untersuchungen haben das Potenzial der AR-Technologie und ihre transformativen Auswirkungen in verschiedenen Bereichen wie Bildung, Gesundheitswesen, Unterhaltung und Militär usw. untersucht.

Die AR-Technologie hat Perspektiven für die Verbesserung der menschlichen Erfahrungen und die Steigerung der Effizienz eröffnet. Im Bildungswesen zum Beispiel hat sich AR bei der Erstellung interaktiver Lernmaterialien bewährt, die es den Schülern ermöglichen, komplexe Konzepte besser zu verstehen. Im Gesundheitswesen wird AR eingesetzt, um die Ergebnisse zu verbessern, indem Chirurgen während eines Eingriffs in Echtzeit Informationen und Anleitungen erhalten. Das Militär nutzt AR zur Ausbildung von Soldaten durch Simulation von Einsatzszenarien, um die Bereitschaft für reale Situationen zu verbessern. Darüber hinaus nutzt die Unterhaltungsindustrie AR, um lebensechte Erfahrungen für die Nutzer zu schaffen.

Während die Vorteile der AR-Technologie beträchtlich sind, gibt es auch Herausforderungen und Einschränkungen, die berücksichtigt werden müssen. Ein grosses Hindernis sind die mit der Entwicklung von AR-Anwendungen verbundenen Kosten - ein Hindernis, das viele Unternehmen davon abhalten könnte, diese Technologie vollständig zu übernehmen. Ein weiteres Problem sind die Hardware- und Softwareanforderungen, die die AR-Technologie für bestimmte Bevölkerungsgruppen weniger zugänglich machen.

Trotz dieser Schwierigkeiten ist es wichtig, das Potenzial von Augmented Reality zu erkennen, das unsere Art der Interaktion mit der Welt verändern wird. Mit den Fortschritten in der AR-Technologie und ihrer breiten Anwendung können wir für die Zukunft noch mehr bahnbrechende Anwendungen und Erfindungen erwarten. Es ist klar, dass AR nicht nur ein Trend ist, sondern ein hervorragendes Werkzeug, das zahlreiche Bereiche revolutionieren und das Wohlbefinden der Menschen weltweit verbessern kann.

    \newpage

    \topskip0pt
    \vspace*{\fill}
    „Hiermit versichere ich, dass ich die vorstehende Arbeit selbstständig angefertigt habe. Alle Stellen, die wörtlich oder sinngemäss übernommen wurden, habe ich als solche gekennzeichnet.“
    \begin{flushright}
        Zhabrovets Andrii
    \end{flushright}
    \vspace*{\fill}

    
    % Here we import our bibliography(citations) to the document 
    \newpage

    \clearpage

    \bibliography{content/bibliography}



    \newpage
    \addcontentsline{toc}{section}{\listfigurename} % Fuege Abb.vz zu TOC hinzu
    \listoffigures


\end{document}